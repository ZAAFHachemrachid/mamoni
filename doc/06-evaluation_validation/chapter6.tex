\chapter{Évaluation et Validation}

\section{Métriques d'Évaluation}

\subsection{Métriques de Performance}
Les performances du système sont évaluées selon plusieurs critères :

\begin{itemize}
    \item \textbf{Précision (Accuracy)}
    \begin{equation}
        Accuracy = \frac{\text{Nombre de prédictions correctes}}{\text{Nombre total de prédictions}}
    \end{equation}

    \item \textbf{Matrice de Confusion}
    \begin{itemize}
        \item Vrais positifs (VP)
        \item Faux positifs (FP)
        \item Vrais négatifs (VN)
        \item Faux négatifs (FN)
    \end{itemize}

    \item \textbf{Métriques par Classe}
    \begin{align*}
        Précision &= \frac{VP}{VP + FP} \\
        Rappel &= \frac{VP}{VP + FN} \\
        F1\text{-}score &= 2 \times \frac{Précision \times Rappel}{Précision + Rappel}
    \end{align*}
\end{itemize}

\subsection{Métriques de Clustering}
Évaluation spécifique aux k-moyennes :

\begin{itemize}
    \item \textbf{Inertie}
    \begin{equation}
        Inertie = \sum_{i=1}^{n} \min_{\mu_j \in C} (||x_i - \mu_j||^2)
    \end{equation}

    \item \textbf{Score de Silhouette}
    \begin{equation}
        s(i) = \frac{b(i) - a(i)}{\max(a(i), b(i))}
    \end{equation}
    où :
    \begin{itemize}
        \item $a(i)$ : distance moyenne intra-cluster
        \item $b(i)$ : distance moyenne au cluster le plus proche
    \end{itemize}

    \item \textbf{Pureté des Clusters}
    \begin{equation}
        Pureté = \frac{1}{N} \sum_{i=1}^{k} \max_j |c_i \cap t_j|
    \end{equation}
\end{itemize}

\section{Validation Croisée}

\subsection{Protocole de Validation}
La validation suit un processus rigoureux :

\begin{enumerate}
    \item \textbf{K-Fold Cross Validation}
    \begin{itemize}
        \item Division en k sous-ensembles
        \item Rotation des ensembles test/validation
        \item Agrégation des résultats
    \end{itemize}

    \item \textbf{Stratification}
    \begin{itemize}
        \item Maintien de la distribution des classes
        \item Équilibrage des ensembles
        \item Représentativité des échantillons
    \end{itemize}
\end{enumerate}

\subsection{Analyse des Résultats}
L'analyse comprend plusieurs aspects :

\begin{itemize}
    \item \textbf{Statistiques Descriptives}
    \begin{itemize}
        \item Moyenne et écart-type des performances
        \item Intervalles de confiance
        \item Tests de significativité
    \end{itemize}

    \item \textbf{Analyse des Erreurs}
    \begin{itemize}
        \item Identification des cas problématiques
        \item Patterns d'erreurs récurrents
        \item Suggestions d'amélioration
    \end{itemize}

    \item \textbf{Courbes de Performance}
    \begin{itemize}
        \item Courbes d'apprentissage
        \item Courbes ROC
        \item Matrices de confusion normalisées
    \end{itemize}
\end{itemize}

\subsection{Robustesse du Modèle}
Évaluation de la stabilité :

\begin{itemize}
    \item \textbf{Tests de Sensibilité}
    \begin{itemize}
        \item Variation des paramètres
        \item Perturbations des données
        \item Analyse de stabilité
    \end{itemize}

    \item \textbf{Validation sur Données Indépendantes}
    \begin{itemize}
        \item Jeux de données externes
        \item Conditions réelles d'utilisation
        \item Généralisation des performances
    \end{itemize}
\end{itemize}