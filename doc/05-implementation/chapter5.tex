\chapter{Implémentation}

\section{Environnement de Développement}

\subsection{Outils et Technologies}
L'environnement de développement a été soigneusement sélectionné :

\begin{itemize}
    \item \textbf{Langage de Programmation}
    \begin{itemize}
        \item Python 3.8+
        \item Bibliothèques scientifiques (NumPy, SciPy)
        \item Outils de visualisation (Matplotlib, Seaborn)
    \end{itemize}

    \item \textbf{Frameworks et Bibliothèques}
    \begin{itemize}
        \item scikit-learn pour l'implémentation des k-moyennes
        \item Pandas pour la manipulation des données
        \item OpenCV pour le traitement d'images
    \end{itemize}

    \item \textbf{Environnement d'Exécution}
    \begin{itemize}
        \item Jupyter Notebooks pour le prototypage
        \item Scripts Python modulaires pour la production
        \item Tests unitaires avec pytest
    \end{itemize}
\end{itemize}

\section{Code et Algorithmes}

\subsection{Structure du Code}
L'architecture du projet suit une organisation modulaire :

\begin{verbatim}
project/
├── src/
│   ├── preprocessing/
│   ├── kmeans/
│   ├── evaluation/
│   └── utils/
├── tests/
├── data/
└── notebooks/
\end{verbatim}

\subsection{Implémentation des Algorithmes}
Les principales composantes algorithmiques comprennent :

\begin{enumerate}
    \item \textbf{Prétraitement}
    \begin{lstlisting}[language=Python]
def preprocess_image(image):
    # Normalisation
    normalized = image.astype(float) / 255.0
    # Redimensionnement
    resized = cv2.resize(normalized, (28, 28))
    return resized
    \end{lstlisting}

    \item \textbf{K-moyennes}
    \begin{lstlisting}[language=Python]
class KMeansClassifier:
    def __init__(self, n_clusters):
        self.n_clusters = n_clusters
        self.centroids = None

    def fit(self, X):
        # Initialisation k-means++
        self.centroids = self._initialize_centroids(X)
        while not self._converged():
            # Assignation
            labels = self._assign_clusters(X)
            # Mise à jour
            self._update_centroids(X, labels)
    \end{lstlisting}
\end{enumerate}

\subsection{Optimisations Techniques}
Plusieurs optimisations ont été implémentées :

\begin{itemize}
    \item \textbf{Vectorisation}
    \begin{itemize}
        \item Utilisation des opérations NumPy
        \item Minimisation des boucles Python
        \item Calculs matriciels optimisés
    \end{itemize}

    \item \textbf{Gestion de la Mémoire}
    \begin{itemize}
        \item Traitement par lots
        \item Libération proactive de la mémoire
        \item Optimisation des structures de données
    \end{itemize}

    \item \textbf{Parallélisation}
    \begin{itemize}
        \item Utilisation de multiprocessing
        \item Calculs distribués des distances
        \item Parallélisation des validations croisées
    \end{itemize}
\end{itemize}

\subsection{Tests et Validation}
Un ensemble complet de tests a été mis en place :

\begin{itemize}
    \item \textbf{Tests Unitaires}
    \begin{itemize}
        \item Validation des fonctions individuelles
        \item Tests des cas limites
        \item Vérification des sorties attendues
    \end{itemize}

    \item \textbf{Tests d'Intégration}
    \begin{itemize}
        \item Validation du pipeline complet
        \item Tests de performance
        \item Vérification de la cohérence
    \end{itemize}
\end{itemize}