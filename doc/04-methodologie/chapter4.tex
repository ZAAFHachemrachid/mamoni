\chapter{Méthodologie}

\section{Initialisation des Groupes}

\subsection{Choix du Nombre de Groupes}
La détermination du nombre optimal de groupes (k) est cruciale pour l'algorithme des k-moyennes :

\begin{itemize}
    \item \textbf{Méthodes de Sélection}
    \begin{itemize}
        \item Méthode du coude (Elbow method)
        \item Silhouette score
        \item Gap statistic
    \end{itemize}

    \item \textbf{Considérations Pratiques}
    \begin{itemize}
        \item Complexité computationnelle
        \item Compromis biais-variance
        \item Interprétabilité des résultats
    \end{itemize}
\end{itemize}

\subsection{Stratégies d'Initialisation}
Plusieurs approches sont considérées pour l'initialisation des centroïdes :

\begin{enumerate}
    \item \textbf{K-means++}
    \begin{itemize}
        \item Sélection probabiliste des centroïdes initiaux
        \item Maximisation des distances entre centroïdes
        \item Réduction de la sensibilité à l'initialisation
    \end{itemize}

    \item \textbf{Initialisation Aléatoire}
    \begin{itemize}
        \item Sélection aléatoire des points
        \item Multiples initialisations
        \item Sélection de la meilleure configuration
    \end{itemize}
\end{enumerate}

\section{Application de l'Algorithme k-Moyennes}

\subsection{Processus Itératif}
L'algorithme suit un processus itératif en deux étapes :

\begin{enumerate}
    \item \textbf{Attribution des Points}
    \begin{itemize}
        \item Calcul des distances aux centroïdes
        \item Assignation au groupe le plus proche
        \item Mise à jour des appartenances
    \end{itemize}

    \item \textbf{Mise à jour des Centroïdes}
    \begin{itemize}
        \item Calcul des moyennes par groupe
        \item Déplacement des centroïdes
        \item Vérification de la convergence
    \end{itemize}
\end{enumerate}

\subsection{Critères de Convergence}
Les critères d'arrêt de l'algorithme incluent :

\begin{itemize}
    \item \textbf{Conditions d'Arrêt}
    \begin{itemize}
        \item Nombre maximum d'itérations
        \item Seuil de variation minimal
        \item Stabilité des assignations
    \end{itemize}

    \item \textbf{Métriques de Qualité}
    \begin{itemize}
        \item Inertie intra-cluster
        \item Séparation inter-clusters
        \item Homogénéité des groupes
    \end{itemize}
\end{itemize}

\subsection{Optimisations}
Plusieurs optimisations sont mises en œuvre :

\begin{itemize}
    \item \textbf{Algorithmes Efficaces}
    \begin{itemize}
        \item Structures de données optimisées
        \item Calculs vectorisés
        \item Parallélisation des opérations
    \end{itemize}

    \item \textbf{Gestion de la Mémoire}
    \begin{itemize}
        \item Traitement par lots
        \item Mise en cache intelligente
        \item Optimisation des ressources
    \end{itemize}
\end{itemize}