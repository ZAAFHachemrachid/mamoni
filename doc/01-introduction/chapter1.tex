\chapter{Introduction}

\section{Contexte et Problématique}

La reconnaissance de chiffres manuscrits constitue l'un des défis fondamentaux dans le domaine de l'intelligence artificielle et de la vision par ordinateur. Ce problème, en apparence simple pour l'œil humain, représente un défi considérable pour les systèmes automatisés.

\subsection{Présentation du problème}
La variabilité inhérente à l'écriture manuscrite, incluant les différents styles d'écriture, l'inclinaison, l'épaisseur des traits et la qualité de l'écriture, rend la tâche de reconnaissance particulièrement complexe. Chaque individu possède sa propre façon d'écrire les chiffres, créant ainsi une diversité importante dans les formes à reconnaître.

\subsection{Importance et applications}
Les applications de la reconnaissance de chiffres manuscrits sont nombreuses et variées :
\begin{itemize}
    \item Traitement automatique des formulaires administratifs
    \item Lecture automatique des codes postaux
    \item Numérisation de documents historiques
    \item Systèmes bancaires pour la lecture des chèques
    \item Applications éducatives
\end{itemize}

\section{Objectifs du Projet}

\subsection{Développement d'un système efficace}
Notre objectif principal est de développer un système robuste capable de reconnaître les chiffres manuscrits avec une haute précision. Ce système doit être :
\begin{itemize}
    \item Rapide dans son temps de traitement
    \item Précis dans ses prédictions
    \item Adaptable à différents styles d'écriture
\end{itemize}

\subsection{Évaluation de la méthode des k-moyennes}
Un objectif spécifique de ce projet est d'évaluer l'efficacité de l'algorithme des k-moyennes pour cette tâche. Nous nous concentrerons sur :
\begin{itemize}
    \item L'analyse de la performance de classification
    \item L'optimisation des paramètres de l'algorithme
    \item La comparaison avec d'autres méthodes existantes
    \item L'évaluation de la robustesse face à différents types d'échantillons
\end{itemize}