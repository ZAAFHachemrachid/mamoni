\chapter{Données et Prétraitement}

\section{Choix de la Base de Données}

\subsection{Base de Données MNIST}
La base de données MNIST (Modified National Institute of Standards and Technology) constitue une référence standard dans le domaine de la reconnaissance de chiffres manuscrits. Elle présente les caractéristiques suivantes :

\begin{itemize}
    \item 60 000 images d'entraînement
    \item 10 000 images de test
    \item Images en niveaux de gris de 28x28 pixels
    \item Chiffres normalisés et centrés
\end{itemize}

\subsection{Caractéristiques des Données}
Les données présentent plusieurs aspects importants :
\begin{itemize}
    \item \textbf{Uniformité} : Taille et format standardisés
    \item \textbf{Diversité} : Différents styles d'écriture
    \item \textbf{Équilibre} : Distribution relativement équitable des classes
    \item \textbf{Qualité} : Images nettoyées et validées
\end{itemize}

\section{Prétraitement des Données}

\subsection{Étapes de Prétraitement}
Le prétraitement des données comprend plusieurs étapes essentielles :

\begin{enumerate}
    \item \textbf{Normalisation}
    \begin{itemize}
        \item Mise à l'échelle des valeurs de pixels [0,1]
        \item Centrage des données
        \item Standardisation
    \end{itemize}

    \item \textbf{Augmentation des Données}
    \begin{itemize}
        \item Rotations légères
        \item Translations
        \item Modifications d'échelle
    \end{itemize}

    \item \textbf{Réduction de Dimension}
    \begin{itemize}
        \item Analyse en Composantes Principales (ACP)
        \item Sélection des caractéristiques principales
        \item Conservation de l'information pertinente
    \end{itemize}
\end{enumerate}

\subsection{Validation de la Qualité}
Pour assurer la qualité du prétraitement :

\begin{itemize}
    \item \textbf{Vérification visuelle}
    \begin{itemize}
        \item Inspection des images transformées
        \item Confirmation de la préservation des caractéristiques importantes
    \end{itemize}

    \item \textbf{Analyses Statistiques}
    \begin{itemize}
        \item Distribution des valeurs de pixels
        \item Variance expliquée après réduction
        \item Corrélations entre caractéristiques
    \end{itemize}

    \item \textbf{Tests de Robustesse}
    \begin{itemize}
        \item Validation sur sous-ensembles
        \item Tests de sensibilité aux paramètres
        \item Évaluation de la stabilité
    \end{itemize}
\end{itemize}